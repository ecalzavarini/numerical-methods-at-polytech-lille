
% Default to the notebook output style

    


% Inherit from the specified cell style.




    
\documentclass{article}

    
    
    \usepackage{graphicx} % Used to insert images
    \usepackage{adjustbox} % Used to constrain images to a maximum size 
    \usepackage{color} % Allow colors to be defined
    \usepackage{enumerate} % Needed for markdown enumerations to work
    \usepackage{geometry} % Used to adjust the document margins
    \usepackage{amsmath} % Equations
    \usepackage{amssymb} % Equations
    \usepackage[mathletters]{ucs} % Extended unicode (utf-8) support
    \usepackage[utf8x]{inputenc} % Allow utf-8 characters in the tex document
    \usepackage{fancyvrb} % verbatim replacement that allows latex
    \usepackage{grffile} % extends the file name processing of package graphics 
                         % to support a larger range 
    % The hyperref package gives us a pdf with properly built
    % internal navigation ('pdf bookmarks' for the table of contents,
    % internal cross-reference links, web links for URLs, etc.)
    \usepackage{hyperref}
    \usepackage{longtable} % longtable support required by pandoc >1.10
    \usepackage{booktabs}  % table support for pandoc > 1.12.2
    

    
    
    \definecolor{orange}{cmyk}{0,0.4,0.8,0.2}
    \definecolor{darkorange}{rgb}{.71,0.21,0.01}
    \definecolor{darkgreen}{rgb}{.12,.54,.11}
    \definecolor{myteal}{rgb}{.26, .44, .56}
    \definecolor{gray}{gray}{0.45}
    \definecolor{lightgray}{gray}{.95}
    \definecolor{mediumgray}{gray}{.8}
    \definecolor{inputbackground}{rgb}{.95, .95, .85}
    \definecolor{outputbackground}{rgb}{.95, .95, .95}
    \definecolor{traceback}{rgb}{1, .95, .95}
    % ansi colors
    \definecolor{red}{rgb}{.6,0,0}
    \definecolor{green}{rgb}{0,.65,0}
    \definecolor{brown}{rgb}{0.6,0.6,0}
    \definecolor{blue}{rgb}{0,.145,.698}
    \definecolor{purple}{rgb}{.698,.145,.698}
    \definecolor{cyan}{rgb}{0,.698,.698}
    \definecolor{lightgray}{gray}{0.5}
    
    % bright ansi colors
    \definecolor{darkgray}{gray}{0.25}
    \definecolor{lightred}{rgb}{1.0,0.39,0.28}
    \definecolor{lightgreen}{rgb}{0.48,0.99,0.0}
    \definecolor{lightblue}{rgb}{0.53,0.81,0.92}
    \definecolor{lightpurple}{rgb}{0.87,0.63,0.87}
    \definecolor{lightcyan}{rgb}{0.5,1.0,0.83}
    
    % commands and environments needed by pandoc snippets
    % extracted from the output of `pandoc -s`
    \DefineVerbatimEnvironment{Highlighting}{Verbatim}{commandchars=\\\{\}}
    % Add ',fontsize=\small' for more characters per line
    \newenvironment{Shaded}{}{}
    \newcommand{\KeywordTok}[1]{\textcolor[rgb]{0.00,0.44,0.13}{\textbf{{#1}}}}
    \newcommand{\DataTypeTok}[1]{\textcolor[rgb]{0.56,0.13,0.00}{{#1}}}
    \newcommand{\DecValTok}[1]{\textcolor[rgb]{0.25,0.63,0.44}{{#1}}}
    \newcommand{\BaseNTok}[1]{\textcolor[rgb]{0.25,0.63,0.44}{{#1}}}
    \newcommand{\FloatTok}[1]{\textcolor[rgb]{0.25,0.63,0.44}{{#1}}}
    \newcommand{\CharTok}[1]{\textcolor[rgb]{0.25,0.44,0.63}{{#1}}}
    \newcommand{\StringTok}[1]{\textcolor[rgb]{0.25,0.44,0.63}{{#1}}}
    \newcommand{\CommentTok}[1]{\textcolor[rgb]{0.38,0.63,0.69}{\textit{{#1}}}}
    \newcommand{\OtherTok}[1]{\textcolor[rgb]{0.00,0.44,0.13}{{#1}}}
    \newcommand{\AlertTok}[1]{\textcolor[rgb]{1.00,0.00,0.00}{\textbf{{#1}}}}
    \newcommand{\FunctionTok}[1]{\textcolor[rgb]{0.02,0.16,0.49}{{#1}}}
    \newcommand{\RegionMarkerTok}[1]{{#1}}
    \newcommand{\ErrorTok}[1]{\textcolor[rgb]{1.00,0.00,0.00}{\textbf{{#1}}}}
    \newcommand{\NormalTok}[1]{{#1}}
    
    % Define a nice break command that doesn't care if a line doesn't already
    % exist.
    \def\br{\hspace*{\fill} \\* }
    % Math Jax compatability definitions
    \def\gt{>}
    \def\lt{<}
    % Document parameters
    \title{MNI-TP3-2015}
    
    
    

    % Pygments definitions
    
\makeatletter
\def\PY@reset{\let\PY@it=\relax \let\PY@bf=\relax%
    \let\PY@ul=\relax \let\PY@tc=\relax%
    \let\PY@bc=\relax \let\PY@ff=\relax}
\def\PY@tok#1{\csname PY@tok@#1\endcsname}
\def\PY@toks#1+{\ifx\relax#1\empty\else%
    \PY@tok{#1}\expandafter\PY@toks\fi}
\def\PY@do#1{\PY@bc{\PY@tc{\PY@ul{%
    \PY@it{\PY@bf{\PY@ff{#1}}}}}}}
\def\PY#1#2{\PY@reset\PY@toks#1+\relax+\PY@do{#2}}

\expandafter\def\csname PY@tok@gd\endcsname{\def\PY@tc##1{\textcolor[rgb]{0.63,0.00,0.00}{##1}}}
\expandafter\def\csname PY@tok@gu\endcsname{\let\PY@bf=\textbf\def\PY@tc##1{\textcolor[rgb]{0.50,0.00,0.50}{##1}}}
\expandafter\def\csname PY@tok@gt\endcsname{\def\PY@tc##1{\textcolor[rgb]{0.00,0.27,0.87}{##1}}}
\expandafter\def\csname PY@tok@gs\endcsname{\let\PY@bf=\textbf}
\expandafter\def\csname PY@tok@gr\endcsname{\def\PY@tc##1{\textcolor[rgb]{1.00,0.00,0.00}{##1}}}
\expandafter\def\csname PY@tok@cm\endcsname{\let\PY@it=\textit\def\PY@tc##1{\textcolor[rgb]{0.25,0.50,0.50}{##1}}}
\expandafter\def\csname PY@tok@vg\endcsname{\def\PY@tc##1{\textcolor[rgb]{0.10,0.09,0.49}{##1}}}
\expandafter\def\csname PY@tok@m\endcsname{\def\PY@tc##1{\textcolor[rgb]{0.40,0.40,0.40}{##1}}}
\expandafter\def\csname PY@tok@mh\endcsname{\def\PY@tc##1{\textcolor[rgb]{0.40,0.40,0.40}{##1}}}
\expandafter\def\csname PY@tok@go\endcsname{\def\PY@tc##1{\textcolor[rgb]{0.53,0.53,0.53}{##1}}}
\expandafter\def\csname PY@tok@ge\endcsname{\let\PY@it=\textit}
\expandafter\def\csname PY@tok@vc\endcsname{\def\PY@tc##1{\textcolor[rgb]{0.10,0.09,0.49}{##1}}}
\expandafter\def\csname PY@tok@il\endcsname{\def\PY@tc##1{\textcolor[rgb]{0.40,0.40,0.40}{##1}}}
\expandafter\def\csname PY@tok@cs\endcsname{\let\PY@it=\textit\def\PY@tc##1{\textcolor[rgb]{0.25,0.50,0.50}{##1}}}
\expandafter\def\csname PY@tok@cp\endcsname{\def\PY@tc##1{\textcolor[rgb]{0.74,0.48,0.00}{##1}}}
\expandafter\def\csname PY@tok@gi\endcsname{\def\PY@tc##1{\textcolor[rgb]{0.00,0.63,0.00}{##1}}}
\expandafter\def\csname PY@tok@gh\endcsname{\let\PY@bf=\textbf\def\PY@tc##1{\textcolor[rgb]{0.00,0.00,0.50}{##1}}}
\expandafter\def\csname PY@tok@ni\endcsname{\let\PY@bf=\textbf\def\PY@tc##1{\textcolor[rgb]{0.60,0.60,0.60}{##1}}}
\expandafter\def\csname PY@tok@nl\endcsname{\def\PY@tc##1{\textcolor[rgb]{0.63,0.63,0.00}{##1}}}
\expandafter\def\csname PY@tok@nn\endcsname{\let\PY@bf=\textbf\def\PY@tc##1{\textcolor[rgb]{0.00,0.00,1.00}{##1}}}
\expandafter\def\csname PY@tok@no\endcsname{\def\PY@tc##1{\textcolor[rgb]{0.53,0.00,0.00}{##1}}}
\expandafter\def\csname PY@tok@na\endcsname{\def\PY@tc##1{\textcolor[rgb]{0.49,0.56,0.16}{##1}}}
\expandafter\def\csname PY@tok@nb\endcsname{\def\PY@tc##1{\textcolor[rgb]{0.00,0.50,0.00}{##1}}}
\expandafter\def\csname PY@tok@nc\endcsname{\let\PY@bf=\textbf\def\PY@tc##1{\textcolor[rgb]{0.00,0.00,1.00}{##1}}}
\expandafter\def\csname PY@tok@nd\endcsname{\def\PY@tc##1{\textcolor[rgb]{0.67,0.13,1.00}{##1}}}
\expandafter\def\csname PY@tok@ne\endcsname{\let\PY@bf=\textbf\def\PY@tc##1{\textcolor[rgb]{0.82,0.25,0.23}{##1}}}
\expandafter\def\csname PY@tok@nf\endcsname{\def\PY@tc##1{\textcolor[rgb]{0.00,0.00,1.00}{##1}}}
\expandafter\def\csname PY@tok@si\endcsname{\let\PY@bf=\textbf\def\PY@tc##1{\textcolor[rgb]{0.73,0.40,0.53}{##1}}}
\expandafter\def\csname PY@tok@s2\endcsname{\def\PY@tc##1{\textcolor[rgb]{0.73,0.13,0.13}{##1}}}
\expandafter\def\csname PY@tok@vi\endcsname{\def\PY@tc##1{\textcolor[rgb]{0.10,0.09,0.49}{##1}}}
\expandafter\def\csname PY@tok@nt\endcsname{\let\PY@bf=\textbf\def\PY@tc##1{\textcolor[rgb]{0.00,0.50,0.00}{##1}}}
\expandafter\def\csname PY@tok@nv\endcsname{\def\PY@tc##1{\textcolor[rgb]{0.10,0.09,0.49}{##1}}}
\expandafter\def\csname PY@tok@s1\endcsname{\def\PY@tc##1{\textcolor[rgb]{0.73,0.13,0.13}{##1}}}
\expandafter\def\csname PY@tok@kd\endcsname{\let\PY@bf=\textbf\def\PY@tc##1{\textcolor[rgb]{0.00,0.50,0.00}{##1}}}
\expandafter\def\csname PY@tok@sh\endcsname{\def\PY@tc##1{\textcolor[rgb]{0.73,0.13,0.13}{##1}}}
\expandafter\def\csname PY@tok@sc\endcsname{\def\PY@tc##1{\textcolor[rgb]{0.73,0.13,0.13}{##1}}}
\expandafter\def\csname PY@tok@sx\endcsname{\def\PY@tc##1{\textcolor[rgb]{0.00,0.50,0.00}{##1}}}
\expandafter\def\csname PY@tok@bp\endcsname{\def\PY@tc##1{\textcolor[rgb]{0.00,0.50,0.00}{##1}}}
\expandafter\def\csname PY@tok@c1\endcsname{\let\PY@it=\textit\def\PY@tc##1{\textcolor[rgb]{0.25,0.50,0.50}{##1}}}
\expandafter\def\csname PY@tok@kc\endcsname{\let\PY@bf=\textbf\def\PY@tc##1{\textcolor[rgb]{0.00,0.50,0.00}{##1}}}
\expandafter\def\csname PY@tok@c\endcsname{\let\PY@it=\textit\def\PY@tc##1{\textcolor[rgb]{0.25,0.50,0.50}{##1}}}
\expandafter\def\csname PY@tok@mf\endcsname{\def\PY@tc##1{\textcolor[rgb]{0.40,0.40,0.40}{##1}}}
\expandafter\def\csname PY@tok@err\endcsname{\def\PY@bc##1{\setlength{\fboxsep}{0pt}\fcolorbox[rgb]{1.00,0.00,0.00}{1,1,1}{\strut ##1}}}
\expandafter\def\csname PY@tok@mb\endcsname{\def\PY@tc##1{\textcolor[rgb]{0.40,0.40,0.40}{##1}}}
\expandafter\def\csname PY@tok@ss\endcsname{\def\PY@tc##1{\textcolor[rgb]{0.10,0.09,0.49}{##1}}}
\expandafter\def\csname PY@tok@sr\endcsname{\def\PY@tc##1{\textcolor[rgb]{0.73,0.40,0.53}{##1}}}
\expandafter\def\csname PY@tok@mo\endcsname{\def\PY@tc##1{\textcolor[rgb]{0.40,0.40,0.40}{##1}}}
\expandafter\def\csname PY@tok@kn\endcsname{\let\PY@bf=\textbf\def\PY@tc##1{\textcolor[rgb]{0.00,0.50,0.00}{##1}}}
\expandafter\def\csname PY@tok@mi\endcsname{\def\PY@tc##1{\textcolor[rgb]{0.40,0.40,0.40}{##1}}}
\expandafter\def\csname PY@tok@gp\endcsname{\let\PY@bf=\textbf\def\PY@tc##1{\textcolor[rgb]{0.00,0.00,0.50}{##1}}}
\expandafter\def\csname PY@tok@o\endcsname{\def\PY@tc##1{\textcolor[rgb]{0.40,0.40,0.40}{##1}}}
\expandafter\def\csname PY@tok@kr\endcsname{\let\PY@bf=\textbf\def\PY@tc##1{\textcolor[rgb]{0.00,0.50,0.00}{##1}}}
\expandafter\def\csname PY@tok@s\endcsname{\def\PY@tc##1{\textcolor[rgb]{0.73,0.13,0.13}{##1}}}
\expandafter\def\csname PY@tok@kp\endcsname{\def\PY@tc##1{\textcolor[rgb]{0.00,0.50,0.00}{##1}}}
\expandafter\def\csname PY@tok@w\endcsname{\def\PY@tc##1{\textcolor[rgb]{0.73,0.73,0.73}{##1}}}
\expandafter\def\csname PY@tok@kt\endcsname{\def\PY@tc##1{\textcolor[rgb]{0.69,0.00,0.25}{##1}}}
\expandafter\def\csname PY@tok@ow\endcsname{\let\PY@bf=\textbf\def\PY@tc##1{\textcolor[rgb]{0.67,0.13,1.00}{##1}}}
\expandafter\def\csname PY@tok@sb\endcsname{\def\PY@tc##1{\textcolor[rgb]{0.73,0.13,0.13}{##1}}}
\expandafter\def\csname PY@tok@k\endcsname{\let\PY@bf=\textbf\def\PY@tc##1{\textcolor[rgb]{0.00,0.50,0.00}{##1}}}
\expandafter\def\csname PY@tok@se\endcsname{\let\PY@bf=\textbf\def\PY@tc##1{\textcolor[rgb]{0.73,0.40,0.13}{##1}}}
\expandafter\def\csname PY@tok@sd\endcsname{\let\PY@it=\textit\def\PY@tc##1{\textcolor[rgb]{0.73,0.13,0.13}{##1}}}

\def\PYZbs{\char`\\}
\def\PYZus{\char`\_}
\def\PYZob{\char`\{}
\def\PYZcb{\char`\}}
\def\PYZca{\char`\^}
\def\PYZam{\char`\&}
\def\PYZlt{\char`\<}
\def\PYZgt{\char`\>}
\def\PYZsh{\char`\#}
\def\PYZpc{\char`\%}
\def\PYZdl{\char`\$}
\def\PYZhy{\char`\-}
\def\PYZsq{\char`\'}
\def\PYZdq{\char`\"}
\def\PYZti{\char`\~}
% for compatibility with earlier versions
\def\PYZat{@}
\def\PYZlb{[}
\def\PYZrb{]}
\makeatother


    % Exact colors from NB
    \definecolor{incolor}{rgb}{0.0, 0.0, 0.5}
    \definecolor{outcolor}{rgb}{0.545, 0.0, 0.0}



    
    % Prevent overflowing lines due to hard-to-break entities
    \sloppy 
    % Setup hyperref package
    \hypersetup{
      breaklinks=true,  % so long urls are correctly broken across lines
      colorlinks=true,
      urlcolor=blue,
      linkcolor=darkorange,
      citecolor=darkgreen,
      }
    % Slightly bigger margins than the latex defaults
    
    \geometry{verbose,tmargin=1in,bmargin=1in,lmargin=1in,rmargin=1in}
    
    

    \begin{document}
    
    
    \maketitle
    
    

    

    \subparagraph{TP 3 -- Méthodes Numériques pour l'Ingénieur CM3 -- Mars 2015}



    \section{Systèmes linéaires}


    Le texte de cette session de travaux pratiques est également disponible
ici

http://nbviewer.ipython.org/github/ecalzavarini/numerical-methods-at-polytech-lille/blob/master/MNI-TP3-2015.ipynb


    \subsubsection{Instructions pour ce TP}


    Pendant ce TP vous aurez à écrire plusieurs scripts (nous vous suggérons
de les nommer script1.py , script2.py ,\ldots{})

    Les scripts doivent être accompagnés par un document descriptif unique (
README.txt ). Dans ce fichier, vous devrez décrire le mode de
fonctionnement des scripts et, si besoin, mettre vos commentaires. Merci
d'y écrires aussi vos nomes et prénoms complets.

    Tous les fichiers doivent etre mis dans un dossier appelé TP1-nom1-nom2
et ensuite être compressés dans un fichier TP1-nom1-nom2.tgz .

    Enfin vous allez envoyer ce fichier par email à l'enseignant :

soit Enrico (enrico.calzavarini@polytech-lille.fr) ou Stefano
(stefano.berti@polytech-lille.fr)

    Vous avez une semaine de temps pour compléter le TP, c'est-à-dire que la
date limite pour envoyer vos travaux est 7 jours après la date du TP
courant.


    \subsection{Objectif}


    On se propose de résoudre le système linéaire \(A\ \vec{x}=\vec{b}\) (où
A représente une matrice régulière d'ordre \(n\), \(\vec{x}\) le vecteur
inconnu et \(\vec{b}\) le second membre) par deux méthodes :

\(a\)) la méthode directe de \(\textbf{Gauss}\) (en appliquant la
stratégie du pivot partiel)

\(b\)) la méthode itérative de \(\textbf{Jacobi}\) (avec le critère
d'arrêt \(|| \vec{x}_{k} - \vec{x}_{k+1} || < \epsilon\ \) où \(k\) est
l'indice d'itération).

Pour cela, on écrira un script python (un pour chaque méthode). Au bout
de pouvoir utiliser le même script plusieurs fois, il est pratique de
faire en sorte que la matrice \(A\) et le vecteur \(\vec{b}\) puissent
être fournis par l'utilisateur.


    \subsection{Programmation et validation}


    On testera le programme sur les systèmes ci-dessous : \[
{A}= 
\left( \begin{array}{cccc}
6 & 2 & 2 & 4 \\
2 & 8 & 2 & 1 \\
4 & 2 & 16 & 8\\
2 & 4 & 1 & 9
\end{array} \right)
\quad 
\vec{b} =
\left( \begin{array}{c}
1 \\
2 \\
4 \\
1 \\
\end{array} \right)
\]

\[
{A}= 
\left( \begin{array}{ccc}
19 & 5 & 7 \\
2 & 7 & 2 \\
1 & 6 & 11 
\end{array} \right)
\quad 
\vec{b} =
\left( \begin{array}{c}
8 \\
2 \\
1 \\
\end{array} \right)
\]

\[
{A}= 
\left( \begin{array}{cccc}
7 & 0 & 1 & -1 \\
3 & 10 & 5 & 0 \\
0 & 4 & 9 & 2\\
1 & -5 & 3 & 15
\end{array} \right)
\quad 
\vec{b} =
\left( \begin{array}{c}
4 \\
3 \\
-3 \\
12 \\
\end{array} \right)
\]

Dans l'application de la méthode de Jacobi, on étudiera l'influence de
la précision \(\epsilon\) sur la solution (et le nombre d'itérations
nécessaires pour l'obtenir).

\(\textbf{Remarque}\) : Avant d'appliquer la méthode de Jacobi, on fera
le test de la diagonale dominante de la matrice \(A\). Le vecteur
initial sera entré par l'utilisateur (nous vous suggérons d'essayer le
vecteur où toutes les composantes sont égales à l'unité :
\([1,1,1,\ldots]\)).

Formuler une conclusion en comparant les résultats obtenus par les deux
méthodes.


    \subsection{Application}


    On dispose de \(n = 7\) points expérimentaux de coordonnées
\((x_i,y_i)\) (\(i=1,...,n\)) suivantes: \[
\begin{array}{l}
(-1.5, 0.9)\\
(-1.0, 1.2)\\ 
(-0.5, -.08)\\ 
(0.0, -2.0)\\ 
(0.5, -1.3)\\
(1.0, -0.5)\\
(1.3, 0.5)
\end{array} 
\]

En utilisant la méthode de Gauss, trouver les paramètres
\(a_k (k = 0,1,2)\) du modèle parabolique
\(y(x) = a_0 + a_1 x + a_2 x^2\) approximant au mieux les données au
sens des moindres carrés. Les valeurs des paramètres seront déterminées
en résolvant le système linéaire suivant :

\[
\left( \begin{array}{ccc}
n & \Sigma_{i=1}^{n} x_i & \Sigma_{i=1}^{n} x_i^2\\
\Sigma_{i=1}^{n} x_i & \Sigma_{i=1}^{n} x_i^2 & \Sigma_{i=1}^{n} x_i^3 \\
\Sigma_{i=1}^{n} x_i^2 & \Sigma_{i=1}^{n} x_i^3 & \Sigma_{i=1}^{n} x_i^4 
\end{array} \right)
\quad 
\left( \begin{array}{c}
a_0 \\
a_1 \\
a_2
\end{array} \right)
= 
\left( \begin{array}{c}
\Sigma_{i=1}^{n} y_i \\
\Sigma_{i=1}^{n} x_i y_i \\
\Sigma_{i=1}^{n} x_i^2 y_i 
\end{array} \right)
\]

    Les points et la courbe d'approximation obtenue seront visualisés sur un
graphique.


    \paragraph{Rappel de Python}


    Pour définir une matrice ou un vecteur :

    \begin{Verbatim}[commandchars=\\\{\}]
{\color{incolor}In [{\color{incolor}3}]:} \PY{k+kn}{import} \PY{n+nn}{numpy} \PY{k+kn}{as} \PY{n+nn}{np}
        \PY{n}{A} \PY{o}{=} \PY{n}{np}\PY{o}{.}\PY{n}{array}\PY{p}{(}\PY{p}{[}\PY{p}{[}\PY{l+m+mi}{3}\PY{p}{,}\PY{l+m+mi}{1}\PY{p}{]}\PY{p}{,}\PY{p}{[}\PY{o}{\PYZhy{}}\PY{l+m+mi}{1}\PY{p}{,}\PY{l+m+mi}{2}\PY{p}{]}\PY{p}{]} \PY{p}{,}\PY{n+nb}{float}\PY{p}{)}
        \PY{n}{b} \PY{o}{=} \PY{n}{np}\PY{o}{.}\PY{n}{array}\PY{p}{(}\PY{p}{[}\PY{l+m+mi}{2}\PY{p}{,}\PY{o}{\PYZhy{}}\PY{l+m+mi}{2}\PY{p}{]}\PY{p}{,}\PY{n+nb}{float}\PY{p}{)}
\end{Verbatim}

    Pour entrer des matrices et des vecteurs, ainsi que pour les combiner
dans une seule matrice:

    \begin{Verbatim}[commandchars=\\\{\}]
{\color{incolor}In [{\color{incolor}}]:} \PY{k+kn}{import} \PY{n+nn}{numpy} \PY{k+kn}{as} \PY{n+nn}{np}
       
       \PY{n}{N}\PY{o}{=}\PY{n+nb}{input}\PY{p}{(}\PY{l+s}{\PYZdq{}}\PY{l+s}{N=?}\PY{l+s}{\PYZdq{}}\PY{p}{)} \PY{c}{\PYZsh{} taille du probleme}
       
       \PY{n}{A}\PY{o}{=}\PY{n+nb}{input}\PY{p}{(}\PY{l+s}{\PYZdq{}}\PY{l+s}{A=?}\PY{l+s}{\PYZdq{}}\PY{p}{)} \PY{c}{\PYZsh{} lecture de la matrice A, p.ex. [[1,2],[3,4]]}
       \PY{n}{b}\PY{o}{=}\PY{n+nb}{input}\PY{p}{(}\PY{l+s}{\PYZdq{}}\PY{l+s}{b=?}\PY{l+s}{\PYZdq{}}\PY{p}{)} \PY{c}{\PYZsh{} lecture du vecteur b, p.ex. [0,1]}
       
       \PY{c}{\PYZsh{} on définit A,b comme des arrays numpy}
       \PY{n}{A}\PY{o}{=}\PY{n}{np}\PY{o}{.}\PY{n}{asarray}\PY{p}{(}\PY{n}{A}\PY{p}{,}\PY{n+nb}{float}\PY{p}{)} 
       \PY{n}{b}\PY{o}{=}\PY{n}{np}\PY{o}{.}\PY{n}{asarray}\PY{p}{(}\PY{n}{b}\PY{p}{,}\PY{n+nb}{float}\PY{p}{)}
       
       \PY{c}{\PYZsh{} on ajoute b comme dernière colonne à A pour obtenir la matrice augmentée Ab}
       \PY{n}{Ab}\PY{o}{=}\PY{n}{np}\PY{o}{.}\PY{n}{column\PYZus{}stack}\PY{p}{(}\PY{p}{(}\PY{n}{A}\PY{p}{,}\PY{n}{b}\PY{p}{)}\PY{p}{)} 
       
       \PY{c}{\PYZsh{} on verifie}
       \PY{k}{print}\PY{p}{(}\PY{l+s}{\PYZdq{}}\PY{l+s}{A=}\PY{l+s}{\PYZdq{}}\PY{p}{)}
       \PY{k}{print}\PY{p}{(}\PY{n}{A}\PY{p}{)}
       \PY{k}{print}\PY{p}{(}\PY{l+s}{\PYZdq{}}\PY{l+s}{b=}\PY{l+s}{\PYZdq{}}\PY{p}{)}
       \PY{k}{print}\PY{p}{(}\PY{n}{b}\PY{p}{)}
       \PY{k}{print}\PY{p}{(}\PY{l+s}{\PYZdq{}}\PY{l+s}{Ab=}\PY{l+s}{\PYZdq{}}\PY{p}{)}
       \PY{k}{print}\PY{p}{(}\PY{n}{Ab}\PY{p}{)}
\end{Verbatim}

    Vous pouvez vérifier la solution du système linéaire
\(A \vec{x} = \vec{b}\) en utilisant la fonction
\(\textbf{linalg.solve}\) de la bibliothèque Numpy :

    \begin{Verbatim}[commandchars=\\\{\}]
{\color{incolor}In [{\color{incolor}}]:} \PY{n}{x} \PY{o}{=} \PY{n}{np}\PY{o}{.}\PY{n}{linalg}\PY{o}{.}\PY{n}{solve}\PY{p}{(}\PY{n}{A}\PY{p}{,} \PY{n}{b}\PY{p}{)}
       \PY{k}{print}\PY{p}{(}\PY{n}{x}\PY{p}{)}
\end{Verbatim}

    autres fonctions utiles de Numpy :

    \begin{Verbatim}[commandchars=\\\{\}]
{\color{incolor}In [{\color{incolor}}]:} \PY{n}{np}\PY{o}{.}\PY{n}{dot}\PY{p}{(}\PY{n}{A}\PY{p}{,}\PY{n}{b}\PY{p}{)} \PY{c}{\PYZsh{} produit scalaire matrice vecteur ( c\PYZsq{}est différent de A*b! )}
       
       \PY{n}{np}\PY{o}{.}\PY{n}{sum}\PY{p}{(}\PY{n}{b}\PY{p}{)}   \PY{c}{\PYZsh{} sommation des composantes du vecteur b}
\end{Verbatim}



    % Add a bibliography block to the postdoc
    
    
    
    \end{document}
